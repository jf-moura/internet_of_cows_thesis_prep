%!TEX root = ../template.tex
%%%%%%%%%%%%%%%%%%%%%%%%%%%%%%%%%%%%%%%%%%%%%%%%%%%%%%%%%%%%%%%%%%%
%% chapter1.tex
%% NOVA thesis document file
%%
%% Chapter with introduction
%%%%%%%%%%%%%%%%%%%%%%%%%%%%%%%%%%%%%%%%%%%%%%%%%%%%%%%%%%%%%%%%%%%

\typeout{NT FILE chapter1.tex}%

\chapter{Introduction}
\label{cha:introduction}
This chapter will explore the motivations for the work to be conducted, the underlying
problem that motivates our work, the objectives and expected contributors to achieve during
its development. This Chapter closes with a brief presentation of the structure for the
remainder of the document.

\section{Motivation}
\label{sec:motivation}
In the Coitadinha Farm, located in Alentejo, Portugal, they have over 150 cows,
alongside many other animals, spread throughout thousands of acres. Controlling that many
animals in such a vast terrain is quite a difficult task. In this context. the manager of the
Farm have interest in monitoring the behaviours of cows. Obtaining useful data such as their
daily patterns of movement, where they spend time and, potentially, manage their location
through the use of adjustable virtual fences. From a research prespective from ecology and
conservation, the aforementioned data is also relevant to understand the impact of cows
behaviour on the ecossystem, in particular, vegetation. Unfortunately there are not many
technological solutions to this end, and those that exist are expensive and based on
closed solutions. Furthermore, the region lacks quality cellular service, which complicates
further the use of technological solutions to simplify this task.

These cows are kept separated in herds depending on their ages, which means that the younger
cows are not in the same herd as the older ones. This kind of separation is quite important
for them to coexist. Inside each herd they follow a hierarchical structure, having a leader
that all the other cows in the herd tend to follow.

The Coitadinha Farm currently has physical fences in place to maintain the multiple
herds separated from each other and protected. However, these fences are not very durable,
which lead to an often replacement and potential danger for the cows. In addition, since the
farm covers an immense amount of land, it is reasonably strenuous to locate all cows and make
sure all are healthy and safe.

Having already available some great options of collars that create virtual fences for all
kinds of animals, there are still no alternative that would work for the Coitadinha Farm.
Mainly because of the lack of cellular service available, but also do to the large number
of cattle herds in this production.

Furthermore, it of the upmost importance to be able to change the herds' location considering
the effect in plant communities from cows grazing in the exact same area for long periods
of time, this should be possible to do with minimal effort from a control station in the main
farm building for instance.

% talk about the common points, the drinking places that are 8 and we can eventually add a 
% point of connection

\section{Problem Statement}
\label{sec:problem_statement}
Currently the cattle in farms are separated with physical fences, which needs to be constantly
repaired or even replaced, that culminates in a costly and laborious task.

Furthermore, seeing as most farms have a vast amount of land where their cows are scattered on,
it becomes quite difficult to provide help if a cow is in danger or lost, especially considering
that this information is often obtained too late.

Another challenge arises when the farms do not have network coverage. This creates obstacles
to propagate messages from the fields to the user, even assuming that cows wear devices
capable of monitoring their behaviours and communicating with a centralized infrastructure.

Ultimately, there is no low consuming, reliable and robust collar for tracking wildlife in a
rural area, with no access to network infrastructures, and with the option to create a virtual
fence. We will tacle this problem by proposing a novel and innovative solution.

\section{Objectives}
\label{sec:objectives}
During this dissertation we expect to develop a fully functional prototype of an animal collar,
adaptable to any cow, that connects to a \Gls{GPS} and creates a virtual fence for each herd. This
fence should be adjustable accordingly to the user's desire and the collars should send a
vibration to the cows, if they are located outside the fence area, in order to get them back
inside.

These collars should provide accurate information about the cows' locations as well as be
highly scalable to handle all the cows' data. The ultimate goal is to find a reliable and
not much consuming solution to deal with the sensing information collected and transmit it
back to the users. To minimize cost and deal with the lack of infrastructure we plan to explore
the hability of collars to exchange information directly through the use of gossip protocols.
This will allow to minimize the number of collars that need a \glsxtrshort{GPS} tracker and
more powerful radio devices.

\section{Expected Contributors}
\label{sec:expected_contributors}
%TODO

\section{Document Structure}
\label{sec:structure}
The remainder of this dissertation is organized as follows:
\begin{description}
      \item[Chapter~\ref{cha:related_work}] - Related Work: includes research on existing
            protocols for broadcasting, particularly the Gossip Protocol, presents wireless
            sensor networks and some of its applications, reflects on how cows behave in a
            herd and their habits and lastly introduces a few existing collars and their
            specifications.
      \item[Chapter~\ref{cha:work_plan}] - Work Plan: a description of the future work
            organization and explication of each work phase.
\end{description}
