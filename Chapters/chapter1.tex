%!TEX root = ../template.tex
%%%%%%%%%%%%%%%%%%%%%%%%%%%%%%%%%%%%%%%%%%%%%%%%%%%%%%%%%%%%%%%%%%%
%% chapter1.tex
%% NOVA thesis document file
%%
%% Chapter with introduction
%%%%%%%%%%%%%%%%%%%%%%%%%%%%%%%%%%%%%%%%%%%%%%%%%%%%%%%%%%%%%%%%%%%

\typeout{NT FILE chapter1.tex}%

\chapter{Introduction}
\label{cha:introduction}
This chapter will explore the motivations for this thesis development, the
underlying problem that provoced it, the objectives expected to achive during its
development and finish with the current document structure.

\section{Motivations}
\label{sec:motivations}
In the xxx Farm, located in xxx, they have over 150 cows, alongside many other animals,
spread throughout xxx acres. Controling that many animals in such a vaste terrain is quite a
difficult task. Futhermore, the region lacks cellular service, which complicates this
mission even more.

The cows are kept seperated in herds depending on their ages, this means that the
younger cows are not in the same herd as the older ones. This kind of seperation
is quite important for them to coexist. Inside each herd they follow a hierarchical
structure, having a leader that all the other cows follow.

The xxx Farm currently has physical fences in place to mantain the multiple
herds seperated from each other and protected. However, this fences are
not very persistent, which lead to an often replacement and potencial danger for
the cows. In addition, since the farm has an immensive amount of land, it is
reasonably strenuous to locate all cows and make sure all are healthy and safe.

Having already available some great options of collars that create virtual fences
for all kinds of animals, there are still no alternative that would work for
the xxx Farm. Mainly because of the lack of cellular service
available, but also do to the immensely amount of cattle at this location.

%the effect in plant communities due to cows behaviour


% How to make the communications protocol both effective and power-efficient? To what 
% extent can we rely on ad hoc, peer-to-peer transfers in a sparsely-connected 
% spatially-huge sensor network? And finally, how can we provide comprehensive tracking of 
% a collection of animals, even if some of the animals axe reclusive and rarely are close 
% enough to humans to have their data logs uploaded directly?


\section{Problem Statement}
\label{sec:problem_statement}
Currently the cattle in farms is seperated with physical fences, which needs to be constantly
repaired or even replaced, that culminates in a costly and laborious task. Futhermore, having
most farms a vast amount of land where the cows are scattered on, it becomes quite difficult to
provide help if a cow is in danger or lost, specially considering that this information is
often obtained too late. Another challenge needed to be considered is that some of these farms
do not have cellular service, which makes it harder to propagate messages from the fields to
user.


there is no low consuming, reliable and robust to faults collar for tracking cattle,
with the option to create a virtual fence for the herds, modify its barriers based on the
user's desire and ultimatly change the cows course when they deviate from their path.

\section{Objectives}
\label{sec:objectives}
During this dissertation it is expected to be developed a fully functional animal collar,
adaptable to any cow, that connects to gps and creates a virtual fence for each herd. This
fence should be adjustable accordingly to the user's desire and the collars should send a
vibration to the cows, if they are located ouside the fence area, in order to get them back
inside.

These collars should provide accurate information about the cows locations as well
as be highly scalable to handle all the cows informations. The ultimate goal is to find a
reliable and not much consuming solution to deal with the sensing informations.


% fully-functional, highly-mobile, energy-efficient sensing
% system that determines accurate positional data and can
% propagate it through the network.
% we want to find a reliable and not much consuming solution
% operational costs

\section{Structure}
\label{sec:structure}
The remainder of this dissertation is organized as follows:
\begin{itemize}
      \item Chapter 2 - \nameref{cha:related_work}: includes research on existing
            protocols for broadcasting, focusing on Gossip Protocol, options for wireless
            sensor networks, available sensors and Arduino alternatives, how cows behave
            in a herd and lastly some possible existing collars.
      \item Chapter 3 - \nameref{cha:work_plan}: a description of the future work
            organization and explication of each work phase.
\end{itemize}
