%!TEX root = ../template.tex
%%%%%%%%%%%%%%%%%%%%%%%%%%%%%%%%%%%%%%%%%%%%%%%%%%%%%%%%%%%%%%%%%%%
%% chapter1.tex
%% NOVA thesis document file
%%
%% Chapter with introduction
%%%%%%%%%%%%%%%%%%%%%%%%%%%%%%%%%%%%%%%%%%%%%%%%%%%%%%%%%%%%%%%%%%%

\typeout{NT FILE chapter1.tex}%

\chapter{Introduction}
\label{cha:introduction}
This chapter will explore the motivations for this thesis development, the
underlying problem that provoked it, the objectives expected to achieve during its
development and finish with the current document structure.

\section{Motivations}
\label{sec:motivations}
In the xxx Farm, located in xxx, they have over 150 cows, alongside many other animals, spread
throughout xxx acres. Controlling that many animals in such a vast terrain is quite a difficult
task. Furthermore, the region lacks cellular service, which complicates this mission even more.

These cows are kept separated in herds depending on their ages, which means that the younger
cows are not in the same herd as the older ones. This kind of separation is quite important
for them to coexist. Inside each herd they follow a hierarchical structure, having a leader
that all the other cows follow.

The xxx Farm currently has physical fences in place to maintain the multiple herds separated
from each other and protected. However, these fences are not very persistent, which lead to an
often replacement and potential danger for the cows. In addition, since the farm has an
immense amount of land, it is reasonably strenuous to locate all cows and make sure all
are healthy and safe.

Having already available some great options of collars that create virtual fences for all
kinds of animals, there are still no alternative that would work for the xxx Farm. Mainly
because of the lack of cellular service available, but also do to the immensely amount of
cattle at this location.

Furthermore, it of the upmost importance to be able to change the herds' location considering
the effect in plant communities from cows grazing in the exact same area for long periods
of time.

\section{Problem Statement}
\label{sec:problem_statement}
Currently the cattle in farms are separated with physical fences, which needs to be constantly
repaired or even replaced, that culminates in a costly and laborious task.

Furthermore, seeing as most farms have a vast amount of land where their cows are scattered on,
it becomes quite difficult to provide help if a cow is in danger or lost, especially considering
that this information is often obtained too late.

Another challenge arises when the farms do not have network towers. This creates obstacles to
propagate messages from the fields to user.

Ultimately, there is no low consuming, reliable and robust collar for tracking wildlife in a
rural area, with no access to network infrastructures, and with the option to create a virtual
fence.

\section{Objectives}
\label{sec:objectives}
During this dissertation it is expected to be developed a fully functional animal collar,
adaptable to any cow, that connects to a \Gls{GPS} and creates a virtual fence for each herd. This
fence should be adjustable accordingly to the user's desire and the collars should send a
vibration to the cows, if they are located outside the fence area, in order to get them back
inside.

These collars should provide accurate information about the cows' locations as well as be
highly scalable to handle all the cows' data. The ultimate goal is to find a reliable and
not much consuming solution to deal with the sensing information collected and transmit it
back to the users.


\section{Structure}
\label{sec:structure}
The remainder of this dissertation is organized as follows:
\begin{itemize}
      \item Chapter 2 - \nameref{cha:related_work}: includes research on existing protocols
            for broadcasting, particularly the Gossip Protocol, presents wireless sensor
            networks and some of its applications, reflects on how cows behave in a herd and
            their habits and lastly introduces a few existing collars and their specifications.
      \item Chapter 3 - \nameref{cha:work_plan}: a description of the future work
            organization and explication of each work phase.
\end{itemize}
