%!TEX root = ../template.tex
%%%%%%%%%%%%%%%%%%%%%%%%%%%%%%%%%%%%%%%%%%%%%%%%%%%%%%%%%%%%%%%%%%%
%% chapter1.tex
%% NOVA thesis document file
%%
%% Chapter with introduction
%%%%%%%%%%%%%%%%%%%%%%%%%%%%%%%%%%%%%%%%%%%%%%%%%%%%%%%%%%%%%%%%%%%

\typeout{NT FILE chapter1.tex}%

\chapter{Introduction}
\label{cha:introduction}
This chapter will explore the motivations for this thesis development, the
underlying problem that provoced the need for this dissertation, the objectives
expected to achive during its development and the current document structure.

\section{Motivations}
\label{sec:motivations}
In the Bla Bla Bla Farm, located in Bla Bla Bla, they have over xxx cows,
alongside many other animals, spread throughout xxxacres. Controling that many animals
in such a vaste terrain is quite a difficult task. Futhermore, the region lacks
network connection, which complicates this mission even more.

The cows are kept seperated in herds depending on their ages, this means that the
younger cows are not in the same herd as the older ones. This kind of seperation
is quite important for them to coexist. Inside each herd they follow a hierarchical
structure, having a leader that all the other cows follow.

The BLA BLA BLA Farm currently has physical fences in place to mantain the multiple
herds seperated from each other and protected. However, this fences are
not very persistent, which lead to an often replacement and potencial danger for
the cows. In addition, since the farm has an immensive amount of land, it is
reasonably strenuous to locate all cows and make sure all are healthy and safe.

Having already available some great options of collars that create virtual fences
for all kinds of animals, there are still no alternative that would work for
the BLA BLA BLA Farm. Mainly because of the lack of network connection
available, but also do to the immensely amount of cattle at this location.

%the effect in plant communities due to cows behaviour


\section{Problem Statement}
\label{sec:problem_statement}
% small description

\section{Objectives}
\label{sec:objectives}
During this dissertation it is expected to be developed a fully
functional animal collar, adaptable to any cow, that connects to gps and creates a
virtual fence for each herd. This fence should be adjustable accordingly to the
user's wish and the collar should send a vibration to the cows, when this fence
area is changed, in order to get the herd to the new location created by the new
virtual fence position.

The collars should provide accurate information about the cows locations as well
as be highly scalable to handle all the cows informations.
% we want to find a reliable and not much consuming solution
% operational costs

\section{Structure}
\label{sec:structure}
The remainder of this dissertation is organized as follows:
\begin{itemize}
      \item Chapter 2 - \nameref{cha:related_work}: includes research on existing
            protocols for broadcasting, focusing on Gossip Protocol, options for wireless
            sensor networks, available sensors and Arduino alternatives, how cows behave
            in a herd and lastly some possible existing collars.
      \item Chapter 3 - \nameref{cha:work_plan}: a description of the future work
            organization and explication of each work phase.
\end{itemize}
