%!TEX root = ../template.tex
%%%%%%%%%%%%%%%%%%%%%%%%%%%%%%%%%%%%%%%%%%%%%%%%%%%%%%%%%%%%%%%%%%%
%% chapter2.tex
%% NOVA thesis document file
%%
%% Chapter with the state of the art
%%%%%%%%%%%%%%%%%%%%%%%%%%%%%%%%%%%%%%%%%%%%%%%%%%%%%%%%%%%%%%%%%%%

\typeout{NT FILE chapter2.tex}%

\chapter{State of the Art}
\label{cha:state_of_the_art}

\epigraph{
  This work is licensed under the \href{https://www.latex-project.org/lppl/lppl-1-3c/}{\LaTeX\ Project Public License v1.3c}.
  To view a copy of this license, visit the \href{https://www.latex-project.org/lppl/}{LaTeX project public license}.
}

\section{Gossip Protocol} 
\label{sub:gossip_protocol}

This first Chapter introduces the \gls{novathesis} template and how it is organized. In Chapter~\ref{cha:users_manual} you can find some specific instructions on how to use the \gls{novathesis} template.  Chapter~\ref{cha:a_short_latex_tutorial_with_examples} shows some examples and give some hints on how to write your text. Please read these next Chapters carefully.

\subsection{Your Time is Precious}
\label{sub:time_is_money}

Did you learn how to drive by sitting by the wheel and throwing your car into the road?  Most probably you did take your time \emph{learning the rules} and \emph{practicing} first! Likewise, it is not wise to throw yourself at the task of writing a thesis/dissertation in \LaTeX\ without seriously considering the following statement!