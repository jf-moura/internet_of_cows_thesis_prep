%!TEX root = ../template.tex
%%%%%%%%%%%%%%%%%%%%%%%%%%%%%%%%%%%%%%%%%%%%%%%%%%%%%%%%%%%%%%%%%%%%
%% chapter2.tex
%% NOVA thesis document file
%%
%% Chapter with the template manual
%%%%%%%%%%%%%%%%%%%%%%%%%%%%%%%%%%%%%%%%%%%%%%%%%%%%%%%%%%%%%%%%%%%%

\typeout{NT FILE chapter2.tex}%

\chapter{Related Work}
\label{cha:related_work}
% add some introduction to this chapter

\section{Gossip Protocol}
\label{sec:gossip_protocol}

\subsection{History}
\quad The Gossip Protocol, as the name indicates, was created based on how gossips are
propagated in social groups. In a Gossip Protocol, nodes in a network send the information,
randomly, to other nodes in the same network, similar to how a gossip is spread between
members in a social group.\cite{Leitao07}


\subsection{What is the Gossip Protocol}
\quad The gossip protocol is based on every participant propagating their messages collaboratively
with all the members of their group.

This process starts when a node desires to propagate some piece of information. He will send
his message to t nodes, chosen randomly. When the receiving nodes obtain the message for the
first time, they will do the same as the previous node had done and resend the message to t
randomly chosen nodes. If a node receives the same message twice, it will discard her. However,
since neither node knows who has received each message and who has sent a message to whom, each
node will have to keep a log of all messages that he has already received.

% (Talk about the problems it helped solve)

% not final text, came from chatgpt
It can be effective in distributing information quickly, but it can also be inefficient because
nodes may end up sending the same information to each other multiple times.

\subsection{Strategies}

\subsection{Tree-based Approaches}

\subsection{Examples}

\subsection{Gossip Limitations}

\subsection{Discution}

\section{Wireless Sensor Networks}
\label{sec:wireless_sensor_networks}

\subsection{Definition}

\subsection{Architectures and Strategies}

\subsection{Gossip in WSNs}

\subsection{Applications}
\subsubsection{ZebraNet}
\subsubsection{Wireless Tracking}

\section{Sensors and Arduino}
\label{sec:sensors_and_arduino}

\section{Cows Walking and Eating Habits}
\label{sec:cows}

\section{Existing Collars}
\label{sec:existing_collars}

\section{Summary}
\label{sec:summary}



% not final text, came from chatgpt
% Epidemic protocols, on the other hand, involve nodes sending information to all of their
% neighbors whenever they receive new information. This can be more efficient than gossip
% protocols because nodes only need to send the information once, but it can also be slower
% because it may take longer for the information to reach all nodes in the network.

% \section{Gossip Versus Epidemic Procols}
% \label{sec:gossip_vs_epidemic}
% not final text, came from chatgpt
% Gossip protocols and epidemic protocols are both types of distributed algorithms that are used
% to disseminate information in a network.

% In general, gossip protocols are better suited for situations where it is more important to
% disseminate information quickly, while epidemic protocols are better suited for situations
% where it is more important to ensure that all nodes in the network eventually receive the
% information.