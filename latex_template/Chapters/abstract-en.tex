%!TEX root = ../template.tex
%%%%%%%%%%%%%%%%%%%%%%%%%%%%%%%%%%%%%%%%%%%%%%%%%%%%%%%%%%%%%%%%%%%%
%% abstrac-en.tex
%% NOVA thesis document file
%%
%% Abstract in English([^%]*)
%%%%%%%%%%%%%%%%%%%%%%%%%%%%%%%%%%%%%%%%%%%%%%%%%%%%%%%%%%%%%%%%%%%%

\typeout{NT FILE abstrac-en.tex}%

There have been enormous developments in IoT (Internet of Things) over the past few years,
creating the opportunity to connect multiple devices with each other in diverse conditions.
This thesis focuses on creating an innovative way of connecting cattle and their owners
using wireless ad hoc networks. This will make possible the information collected with a
collar to be transmission in the most remote areas, as it is the case of the Coitadinha
Farm.

For farm owners it is of the upmost importance to be able to know the whereabouts and health
of their cattle throughout all day, as well as maintain the cows separated into herds and
away from dangerous situations.

During this work we plan on building a wireless sensor network using a gossip-based protocol
to disseminate the cattle's localization and other relevant information back to the farm owners.
This information should be relayed using peer-to-peer communication, without the need for an
infrastructure available in the area. We also plan to be able to
maintain the cows inside a virtual fence created by the farm owners. This virtual fence should
be easy to create and relocate as needed.

At the end of our work, we expect to obtain a totally functional collar prototype with an
overall good performance, low cost and low power consumption, able to connect with other collars,
within the same network, without the need of infrastructures available.

\begin{keywords}
    Wireless sensor networks, gossip-based protocols, cattle, sensors, wireless technologies,
    wireless ad hoc networks
\end{keywords}