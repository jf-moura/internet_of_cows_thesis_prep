%!TEX root = ../template.tex
%%%%%%%%%%%%%%%%%%%%%%%%%%%%%%%%%%%%%%%%%%%%%%%%%%%%%%%%%%%%%%%%%%%%
%% abstrac-pt.tex
%% NOVA thesis document file
%%
%% Abstract in Portuguese
%%%%%%%%%%%%%%%%%%%%%%%%%%%%%%%%%%%%%%%%%%%%%%%%%%%%%%%%%%%%%%%%%%%%

\typeout{NT FILE abstrac-pt.tex}%

Durante os últimos anos tem havido grandes desenvolvimentos na área da IoT (Internet of Things),
criando a oportunidade de conectar múltiplos dispositivos uns com os outros, nas mais
diversas condições. Esta dissertação irá se focar na criação de uma forma inovadora de
conectar gado e os seus donos usando redes ad hoc. O que fará possível as informações
recolhidas através de uma coleira serem transmitidas nas áreas mais remotas, como é o caso da
Herdade da Coitadinha.

Para donos de quintas é deveras importante saber a localização e estado de saúde de todas as
suas vacas em qualquer altura do dia, tal como mantê-las separadas em rebanhos e longe de
situações perigosas.

Durante este trabalho planeamos construir uma rede de sensores sem fios utilizando um protocolo
de rumor para a disseminação das localizações dos animais, tal como qualquer outras informações
pertinentes de volta para os donos da quinta. Esta informação deverá ser transmitida utilizando
comunicação peer-to-peer, sem a necessidade de existir uma infraestrutura na área. Também planeamos
ser possível manter as vacas dentro de uma cerca virtual criada pelos donos da quinta. Esta cerca
deverá ser fácil de criação e utilizar, podendo realoca-la conforme a necessidade.

No final do nosso trabalho é expectável obter um protótipo totalmente funcional de uma coleira
com bons valores de performance, preço reduzido e baixo consumo de energia, capaz de conectar-se
com outras coleiras, dentro da mesma rede, sem a necessidade de infraestruturas disponíveis.

\begin{keywords}
    redes de sensores sem fios, protocolos de rumor, gado, sensores, tecnologias sem fios, redes ad hoc sem fios
\end{keywords}